\documentclass{report}
\usepackage[a4paper, total={6in, 8in}]{geometry}
\usepackage{textcomp}
\usepackage[T1]{fontenc}
\usepackage{lmodern}
\usepackage{listings}
\usepackage{xcolor}
\lstset{
    language={[ansi]C},
    basicstyle=\small\sffamily,
    numbers=left,
    numberstyle=\tiny,
    frame=tb,
    tabsize=4,
    columns=fixed,
    showstringspaces=false,
    showtabs=false,
    keepspaces,
    commentstyle=\color{green},
    keywordstyle=\color{blue},
    numberstyle=\color{orange},
    stringstyle=\color{red},
    identifierstyle=\bfseries{\color{black}}
}
\title{libUART - Easy to use UART library \\ Programming Guide v0.1.0.x (rev. 0)}
\author{Johannes Krottmayer}
\begin{document}
\maketitle
\tableofcontents
\chapter{Preamble}
This library is for the easy usage from the UART interface (serial port). It should help
a programmer to easy include UART communications in the main application.
\chapter{Legal Notice}
This library is fully licensed under the ISC license. You can use it for private and
for commercial purposes.
\section{License Text}
Copyright 2025 Johannes Krottmayer <krotti83@proton.me>\newline
\newline
Permission to use, copy, modify, and/or distribute this software for any
purpose with or without fee is hereby granted, provided that the above
copyright notice and this permission notice appear in all copies.\newline
\newline
THE SOFTWARE IS PROVIDED "AS IS" AND THE AUTHOR DISCLAIMS ALL WARRANTIES
WITH REGARD TO THIS SOFTWARE INCLUDING ALL IMPLIED WARRANTIES OF
MERCHANTABILITY AND FITNESS. IN NO EVENT SHALL THE AUTHOR BE LIABLE FOR
ANY SPECIAL, DIRECT, INDIRECT, OR CONSEQUENTIAL DAMAGES OR ANY DAMAGES
WHATSOEVER RESULTING FROM LOSS OF USE, DATA OR PROFITS, WHETHER IN AN
ACTION OF CONTRACT, NEGLIGENCE OR OTHER TORTIOUS ACTION, ARISING OUT OF
OR IN CONNECTION WITH THE USE OR PERFORMANCE OF THIS SOFTWARE.
\chapter{Application Programming Interface (API)}
\section{Header}
For communicating with the library include the following header. It
contains all necessary prototypes and data types like elementary types
and enumerations which are required for the library.
\begin{lstlisting}
#include <UART.h>
\end{lstlisting}
\section{Basic Functions}
This section contains basic functions for working with the library, such
as initialization of the library and opening and closing from an \textbf{UART}
interface.
\subsection{UART\_init() Function}
The \textit{UART\_init() function} initializes the \textbf{UART} library.
This function should always called first before usage of any other
functions with the library.
\subsubsection*{Prototype}
\begin{lstlisting}
#include <UART.h>

void UART_init(void);
\end{lstlisting}
\subsubsection*{Arguments}
None
\subsubsection*{Returns}
None
\subsubsection*{Usage}
\begin{lstlisting}
#include <UART.h>

UART_init();
\end{lstlisting}
\subsection{UART\_open() Function}
The \textit{UART\_open() function} opens an \textbf{UART} interface for
the specific device and configure the \textbf{UART} connection.
\subsubsection*{Prototype}
\begin{lstlisting}
#include <UART.h>

uart_t *UART_open(const char *dev, enum e_baud baud, const char *opt);
\end{lstlisting}
\subsubsection*{Arguments}
\begin{enumerate}
\item Device path and name
\begin{description}
\item[NOTE:] The device name for the \textbf{UART} under the directory
\textit{/dev}. Examples are \textit{/dev/ttyS0} or if the \textbf{UART}
interface is controlled via an \textbf{USB} interface then \textit{/dev/ttyUSB0}.
\end{description}
\item Baud rate
\newline
\newline
\begin{tabular}{| c | c |}
\hline
\multicolumn{2}{|c|}{Available Baud Rates} \\
\hline
Enumeration & Baud rate \\
\hline
UART\_BAUD\_0 & 0 \\
UART\_BAUD\_5 & 5 \\
UART\_BAUD\_75 & 75 \\
UART\_BAUD\_110 & 110 \\
UART\_BAUD\_134 & 134 \\
UART\_BAUD\_150 & 150 \\
UART\_BAUD\_200 & 200 \\
UART\_BAUD\_300 & 300 \\
UART\_BAUD\_600 & 600 \\
UART\_BAUD\_1200 & 1200 \\
UART\_BAUD\_1800 & 1800 \\
UART\_BAUD\_2400 & 2400 \\
UART\_BAUD\_4800 & 4800 \\
UART\_BAUD\_9600 & 9600 \\
UART\_BAUD\_19200 & 19200 \\
UART\_BAUD\_38400 & 38400 \\
UART\_BAUD\_57600 & 57600 \\
UART\_BAUD\_115200 & 115200 \\
UART\_BAUD\_230400 & 230400 \\
UART\_BAUD\_460800 & 460800 \\
UART\_BAUD\_500000 & 500000 \\
UART\_BAUD\_576000 & 576000 \\
UART\_BAUD\_921600 & 921600 \\
UART\_BAUD\_1000000 & 1000000 \\
UART\_BAUD\_1152000 & 1152000 \\
UART\_BAUD\_1500000 & 1500000 \\
UART\_BAUD\_2000000 & 2000000 \\
UART\_BAUD\_2500000 & 2500000 \\
UART\_BAUD\_3000000 & 3000000 \\
UART\_BAUD\_3500000 & 3500000 \\
UART\_BAUD\_4000000 & 4000000 \\
\hline
\end{tabular}
\item Option string
\begin{description}
\item[NOTE:] The option string should contain the number of data bits,
the parity, the number of stop bits and the flow control.
\end{description}
\begin{tabular}{| c | c |}
\hline
\multicolumn{2}{|c|}{Valid Data Bits} \\
\hline
Value & Number of Data Bits \\
\hline
5 & 5 bits \\
6 & 6 bits \\
7 & 7 bits \\
8 & 8 bits \\
\hline
\end{tabular}
\begin{tabular}{| c | c |}
\hline
\multicolumn{2}{|c|}{Valid Parities} \\
\hline
Value & Parity \\
\hline
N & None \\
O & Odd \\
E & Even \\
& \\
\hline
\end{tabular}
\begin{tabular}{| c | c |}
\hline
\multicolumn{2}{|c|}{Valid Stop Bits} \\
\hline
Value & Stop Bits \\
\hline
1 & 1 Stop bit \\
2 & 2 Stop bits \\
& \\
& \\
\hline
\end{tabular}
\begin{tabular}{| c | c |}
\hline
\multicolumn{2}{|c|}{Valid Flow Control} \\
\hline
Value & Flow Control \\
\hline
N & None \\
S & Software \\
H & Hardware \\
& \\
\hline
\end{tabular}
\end{enumerate}
\subsubsection*{Returns}
Returns a valid UART object/handle if success, or \textbf{NULL} on failure.
\subsubsection*{Usage}
\begin{lstlisting}
#include <UART.h>

uart_t *uart;

uart = UART_open("/dev/ttyUSB0", "8N1N");
\end{lstlisting}
\subsection{UART\_close() Function}
The \textit{UART\_close() function} closes the \textbf{UART} interface and
frees the underlying \textbf{UART} object.
\subsubsection*{Prototype}
\begin{lstlisting}
#include <UART.h>

void UART_close(uart_t *uart);
\end{lstlisting}
\subsubsection*{Arguments}
\begin{enumerate}
\item UART object/handle
\begin{description}
\item[NOTE:] Pointer to a valid UART object handle from the type \textit{uart\_t}
created with the \textit{UART\_open() function}.
\end{description}
\end{enumerate}
\subsubsection*{Returns}
None
\subsubsection*{Usage}
\begin{lstlisting}
#include <UART.h>

uart_t *uart;

UART_close(uart);
\end{lstlisting}
\section{Basic Input/Output Functions}
This section contains elementary input/output functions such as
sending and/or receiving data over the \textbf{UART} interface.
\subsection{UART\_send() Function}
The \textit{UART\_send() function} sends a specific amount of
data over the \textbf{UART} interface.
\subsubsection*{Prototype}
\begin{lstlisting}
#include <UART.h>

ssize_t UART_send(uart_t *uart, char *send_buf, size_t len);
\end{lstlisting}
\subsubsection*{Arguments}
\begin{enumerate}
\item UART object/handle
\begin{description}
\item[NOTE:] Pointer to a valid UART object handle from the type \textit{uart\_t}
created with the \textit{UART\_open() function}.
\end{description}
\item Pointer to buffer where the data is stored
\item Number of elements in the buffer (bytes to send)
\end{enumerate}
\subsubsection*{Returns}
Returns the number of sent data (bytes), or \textbf{-1} on failure.
\subsubsection*{Usage}
\begin{lstlisting}
#include <UART.h>

uart_t *uart;
char buf[256];

UART_send(uart, buf, 256);
\end{lstlisting}
\subsection{UART\_recv() Function}
The \textit{UART\_recv() function} receives a specific amount of
data over the \textbf{UART} interface.
\subsubsection*{Prototype}
\begin{lstlisting}
#include <UART.h>

ssize_t UART_recv(uart_t *uart, char *recv_buf, size_t len);
\end{lstlisting}
\subsubsection*{Arguments}
\begin{enumerate}
\item UART object/handle
\begin{description}
\item[NOTE:] Pointer to a valid UART object handle from the type \textit{uart\_t}
created with the \textit{UART\_open() function}.
\end{description}
\item Pointer to buffer where the received data should be stored
\item Number of elements (bytes) to receive
\end{enumerate}
\subsubsection*{Returns}
Returns the number of received data (bytes), or \textbf{-1} on failure.
\subsubsection*{Usage}
\begin{lstlisting}
#include <UART.h>

uart_t *uart;
char buf[256];

UART_recv(uart, buf, 256);
\end{lstlisting}
\section{Input/Output Functions}
This section contains generic input/output functions such as
sending strings and characters and receiving characters over
the \textbf{UART} interface.
\subsection{UART\_puts() Function}
The \textit{UART\_puts() function} sends a string over the
\textbf{UART} interface.
\subsubsection*{Prototype}
\begin{lstlisting}
#include <UART.h>

ssize_t UART_puts(uart_t *uart, char *msg);
\end{lstlisting}
\subsubsection*{Arguments}
\begin{enumerate}
\item UART object/handle
\begin{description}
\item[NOTE:] Pointer to a valid UART object handle from the type \textit{uart\_t}
created with the \textit{UART\_open() function}.
\end{description}
\item Pointer to the string which should be send
\end{enumerate}
\subsubsection*{Returns}
Returns the number of sent data (in bytes), or \textbf{-1} on failure.
\subsubsection*{Usage}
\begin{lstlisting}
#include <UART.h>

UART_puts(uart, "Hello World!\r\n");
\end{lstlisting}
\subsection{UART\_putc() Function}
The \textit{UART\_putc() function} send a single character over
the \textbf{UART} interface.
\subsubsection*{Prototype}
\begin{lstlisting}
#include <UART.h>

int UART_putc(uart_t *uart, char c);
\end{lstlisting}
\subsubsection*{Arguments}
\begin{enumerate}
\item UART object/handle
\begin{description}
\item[NOTE:] Pointer to a valid UART object handle from the type \textit{uart\_t}
created with the \textit{UART\_open() function}.
\end{description}
\item Character to send
\end{enumerate}
\subsubsection*{Returns}
Returns the number of sent data (in bytes), or \textbf{-1} on failure.
\subsubsection*{Usage}
\begin{lstlisting}
#include <UART.h>

UART_putc(uart, 'A');
\end{lstlisting}
\subsection{UART\_getc() Function}
The \textit{UART\_getc() function} receives a single character
over the \textbf{UART} interface.
\subsubsection*{Prototype}
\begin{lstlisting}
#include <UART.h>

int UART_getc(uart_t *uart, char *ret_c);
\end{lstlisting}
\subsubsection*{Arguments}
\begin{enumerate}
\item UART object/handle
\begin{description}
\item[NOTE:] Pointer to a valid UART object handle from the type \textit{uart\_t}
created with the \textit{UART\_open() function}.
\end{description}
\item Pointer to character
\end{enumerate}
\subsubsection*{Returns}
Returns the number of received data (in bytes), or \textbf{-1} on failure.
\subsubsection*{Usage}
\begin{lstlisting}
#include <UART.h>

char c;

UART_getc(uart, &c);
\end{lstlisting}
\subsection{UART\_flush() Function}
The \textit{UART\_flush() function} flushes not sent data (which
resides in the internal buffers) from the \textbf{UART} interface.
\subsubsection*{Prototype}
\begin{lstlisting}
#include <UART.h>

int UART_flush(uart_t *uart);
\end{lstlisting}
\subsubsection*{Arguments}
\begin{enumerate}
\item UART object/handle
\begin{description}
\item[NOTE:] Pointer to a valid UART object handle from the type \textit{uart\_t}
created with the \textit{UART\_open() function}.
\end{description}
\end{enumerate}
\subsubsection*{Returns}
Returns \textbf{0} on success, or \textbf{-1} on failure.
\subsubsection*{Usage}
\begin{lstlisting}
#include <UART.h>

UART_flush(uart);
\end{lstlisting}
\subsection{UART\_set\_pin() Function}
The \textit{UART\_set\_pin() function} changes the pin state for a
specific \textbf{UART} pin from the \textbf{UART} interface.
\subsubsection*{Prototype}
\begin{lstlisting}
#include <UART.h>

int UART_set_pin(uart_t *uart, enum e_pins pin, int state);
\end{lstlisting}
\subsubsection*{Arguments}
\begin{enumerate}
\item UART object/handle
\begin{description}
\item[NOTE:] Pointer to a valid UART object handle from the type \textit{uart\_t}
created with the \textit{UART\_open() function}.
\end{description}
\item UART pin Enumeration
\newline
\newline
\begin{tabular}{| c | c |}
\hline
\multicolumn{2}{|c|}{Valid UART Pins} \\
\hline
Enumeration & Direction \\
\hline
UART\_PIN\_RTS & Output \\
UART\_PIN\_DTR & Output \\
\hline
\end{tabular}
\item UART Pin State
\newline
\newline
\begin{tabular}{| c | c |}
\hline
\multicolumn{2}{|c|}{Valid UART Pin States} \\
\hline
Value & Logic Level \\
\hline
UART\_PIN\_LOW & Low (0) \\
UART\_PIN\_HIGH & High (1) \\
\hline
\end{tabular}
\end{enumerate}
\subsubsection*{Returns}
Returns \textbf{0} on success, or \textbf{-1} on failure.
\subsubsection*{Usage}
\begin{lstlisting}
#include <UART.h>

UART_set_pin(uart, UART_PIN_RTS, UART_PIN_HIGH);
\end{lstlisting}
\subsection{UART\_get\_pin() Function}
The \textit{UART\_get\_pin() function} reads out the pin state for a
specific \textbf{UART} pin from the \textbf{UART} interface.
\subsubsection*{Prototype}
\begin{lstlisting}
#include <UART.h>

int UART_get_pin(uart_t *uart, enum e_pins pin, int *ret_state);
\end{lstlisting}
\subsubsection*{Arguments}
\begin{enumerate}
\item UART object/handle
\begin{description}
\item[NOTE:] Pointer to a valid UART object handle from the type \textit{uart\_t}
created with the \textit{UART\_open() function}.
\end{description}
\item UART pin Enumeration
\newline
\newline
\begin{tabular}{| c | c |}
\hline
\multicolumn{2}{|c|}{Valid UART Pins} \\
\hline
Enumeration & Direction \\
\hline
UART\_PIN\_CTS & Input \\
UART\_PIN\_DSR & Input \\
UART\_PIN\_DCD & Input \\
UART\_PIN\_RI & Input \\
\hline
\end{tabular}
\item Pointer to pin state
\end{enumerate}
\subsubsection*{Returns}
Returns \textbf{0} on success, or \textbf{-1} on failure.
\subsubsection*{Usage}
\begin{lstlisting}
#include <UART.h>

int state;

UART_get_pin(uart, UART_PIN_CTS, &state);
\end{lstlisting}
\section{Configuration Functions}
This section contains generic Configuration functions such as
setting the baud rate and other useful Configuration for the
\textbf{UART} interface.
\subsection{UART\_set\_baud() Function}
The \textit{UART\_set\_baud() function} configures the baud rate
from the \textbf{UART} interface.
\subsubsection*{Prototype}
\begin{lstlisting}
#include <UART.h>

int UART_set_baud(uart_t *uart, enum e_baud baud);
\end{lstlisting}
\subsubsection*{Arguments}
\begin{enumerate}
\item UART object/handle
\begin{description}
\item[NOTE:] Pointer to a valid UART object handle from the type \textit{uart\_t}
created with the \textit{UART\_open() function}.
\end{description}
\item Baud rate
\newline
\newline
\begin{tabular}{| c | c |}
\hline
\multicolumn{2}{|c|}{Available Baud Rates} \\
\hline
Enumeration & Baud rate \\
\hline
UART\_BAUD\_0 & 0 \\
UART\_BAUD\_5 & 5 \\
UART\_BAUD\_75 & 75 \\
UART\_BAUD\_110 & 110 \\
UART\_BAUD\_134 & 134 \\
UART\_BAUD\_150 & 150 \\
UART\_BAUD\_200 & 200 \\
UART\_BAUD\_300 & 300 \\
UART\_BAUD\_600 & 600 \\
UART\_BAUD\_1200 & 1200 \\
UART\_BAUD\_1800 & 1800 \\
UART\_BAUD\_2400 & 2400 \\
UART\_BAUD\_4800 & 4800 \\
UART\_BAUD\_9600 & 9600 \\
UART\_BAUD\_19200 & 19200 \\
UART\_BAUD\_38400 & 38400 \\
UART\_BAUD\_57600 & 57600 \\
UART\_BAUD\_115200 & 115200 \\
UART\_BAUD\_230400 & 230400 \\
UART\_BAUD\_460800 & 460800 \\
UART\_BAUD\_500000 & 500000 \\
UART\_BAUD\_576000 & 576000 \\
UART\_BAUD\_921600 & 921600 \\
UART\_BAUD\_1000000 & 1000000 \\
UART\_BAUD\_1152000 & 1152000 \\
UART\_BAUD\_1500000 & 1500000 \\
UART\_BAUD\_2000000 & 2000000 \\
UART\_BAUD\_2500000 & 2500000 \\
UART\_BAUD\_3000000 & 3000000 \\
UART\_BAUD\_3500000 & 3500000 \\
UART\_BAUD\_4000000 & 4000000 \\
\hline
\end{tabular}
\end{enumerate}
\subsubsection*{Returns}
Returns \textbf{0} on success, or \textbf{-1} on failure.
\subsubsection*{Usage}
\begin{lstlisting}
#include <UART.h>

UART_set_baud(uart, UART_BAUD_115200);
\end{lstlisting}
\subsection{UART\_get\_baud() Function}
The \textit{UART\_get\_baud() function} returns the baud rate
from the \textbf{UART} interface.
\subsubsection*{Prototype}
\begin{lstlisting}
#include <UART.h>

int UART_get_baud(uart_t *uart, int *ret_baud);
\end{lstlisting}
\subsubsection*{Arguments}
\begin{enumerate}
\item UART object/handle
\begin{description}
\item[NOTE:] Pointer to a valid UART object handle from the type \textit{uart\_t}
created with the \textit{UART\_open() function}.
\end{description}
\item Pointer to baud rate
\end{enumerate}
\subsubsection*{Returns}
Returns \textbf{0} on success, or \textbf{-1} on failure.
\subsubsection*{Usage}
\begin{lstlisting}
#include <UART.h>

int baud;

UART_get_baud(uart, &baud);
\end{lstlisting}
\subsection{UART\_set\_databits() Function}
The \textit{UART\_set\_databits() function} configures the number
of data bits from the \textbf{UART} interface.
\subsubsection*{Prototype}
\begin{lstlisting}
#include <UART.h>

int UART_set_databits(uart_t *uart, enum e_data data_bits);
\end{lstlisting}
\subsubsection*{Arguments}
\begin{enumerate}
\item UART object/handle
\begin{description}
\item[NOTE:] Pointer to a valid UART object handle from the type \textit{uart\_t}
created with the \textit{UART\_open() function}.
\end{description}
\item Data Bits Enumeration
\newline
\newline
\begin{tabular}{| c | c |}
\hline
\multicolumn{2}{|c|}{Valid Data Bits} \\
\hline
Enumeration & Number of Data Bits \\
\hline
UART\_DATA\_5 & 5 bits \\
UART\_DATA\_6 & 6 bits \\
UART\_DATA\_7 & 7 bits \\
UART\_DATA\_8 & 8 bits \\
\hline
\end{tabular}
\end{enumerate}
\subsubsection*{Returns}
Returns \textbf{0} on success, or \textbf{-1} on failure.
\subsubsection*{Usage}
\begin{lstlisting}
#include <UART.h>

UART_set_databits(uart, UART_DATA_8);
\end{lstlisting}
\subsection{UART\_get\_databits() Function}
The \textit{UART\_set\_databits() function} returns the number
of data bits from the \textbf{UART} interface.
\subsubsection*{Prototype}
\begin{lstlisting}
#include <UART.h>

int UART_get_databits(uart_t *uart, int *ret_data_bits);
\end{lstlisting}
\subsubsection*{Arguments}
\begin{enumerate}
\item UART object/handle
\begin{description}
\item[NOTE:] Pointer to a valid UART object handle from the type \textit{uart\_t}
created with the \textit{UART\_open() function}.
\end{description}
\item Pointer to data bits
\end{enumerate}
\subsubsection*{Returns}
Returns \textbf{0} on success, or \textbf{-1} on failure.
\subsubsection*{Usage}
\begin{lstlisting}
#include <UART.h>

int data;

UART_get_databits(uart, &data);
\end{lstlisting}
\subsection{UART\_set\_parity() Function}
The \textit{UART\_set\_parity() function} configures the parity
from the \textbf{UART} interface.
\subsubsection*{Prototype}
\begin{lstlisting}
#include <UART.h>

int UART_set_parity(uart_t *uart, enum e_parity parity);
\end{lstlisting}
\subsubsection*{Arguments}
\begin{enumerate}
\item UART object/handle
\begin{description}
\item[NOTE:] Pointer to a valid UART object handle from the type \textit{uart\_t}
created with the \textit{UART\_open() function}.
\end{description}
\item Parity Enumeration
\newline
\newline
\begin{tabular}{| c | c |}
\hline
\multicolumn{2}{|c|}{Valid Parities} \\
\hline
Enumeration & Parity \\
\hline
UART\_PARITY\_NONE & None \\
UART\_PARITY\_ODD & Odd \\
UART\_PARITY\_EVEN & Even \\
\hline
\end{tabular}
\end{enumerate}
\subsubsection*{Returns}
Returns \textbf{0} on success, or \textbf{-1} on failure.
\subsubsection*{Usage}
\begin{lstlisting}
#include <UART.h>

UART_set_parity(uart, UART_PARITY_NONE);
\end{lstlisting}
\subsection{UART\_get\_parity() Function}
The \textit{UART\_get\_parity() function} returns the parity
from the \textbf{UART} interface.
\subsubsection*{Prototype}
\begin{lstlisting}
#include <UART.h>

int UART_get_parity(uart_t *uart, int *ret_parity);
\end{lstlisting}
\subsubsection*{Arguments}
\begin{enumerate}
\item UART object/handle
\begin{description}
\item[NOTE:] Pointer to a valid UART object handle from the type \textit{uart\_t}
created with the \textit{UART\_open() function}.
\end{description}
\item Pointer to parity
\end{enumerate}
\subsubsection*{Returns}
Returns \textbf{0} on success, or \textbf{-1} on failure.
\subsubsection*{Usage}
\begin{lstlisting}
#include <UART.h>

int parity;

UART_get_parity(uart, &parity);
\end{lstlisting}
\subsection{UART\_set\_stopbits() Function}
The \textit{UART\_set\_stopbits() function} configures the number
of stop bits from the \textbf{UART} interface.
\subsubsection*{Prototype}
\begin{lstlisting}
#include <UART.h>

int UART_set_stopbits(uart_t *uart, enum e_stop stop_bits);
\end{lstlisting}
\subsubsection*{Arguments}
\begin{enumerate}
\item UART object/handle
\begin{description}
\item[NOTE:] Pointer to a valid UART object handle from the type \textit{uart\_t}
created with the \textit{UART\_open() function}.
\end{description}
\item Stop bits Enumeration
\newline
\newline
\begin{tabular}{| c | c |}
\hline
\multicolumn{2}{|c|}{Valid Stop Bits} \\
\hline
Enumeration & Stop Bits \\
\hline
UART\_STOP\_1\_0 & 1 Bit \\
UART\_STOP\_2\_0 & 2 Bits \\
\hline
\end{tabular}
\end{enumerate}
\subsubsection*{Returns}
Returns \textbf{0} on success, or \textbf{-1} on failure.
\subsubsection*{Usage}
\begin{lstlisting}
#include <UART.h>

UART_set_stopbits(uart, UART_STOP_1_0);
\end{lstlisting}
\subsection{UART\_get\_stopbits() Function}
The \textit{UART\_get\_stopbits() function} configures the number
of stop bits from the \textbf{UART} interface.
\subsubsection*{Prototype}
\begin{lstlisting}
#include <UART.h>

int UART_get_stopbits(uart_t *uart, int *ret_stop_bits);
\end{lstlisting}
\subsubsection*{Arguments}
\begin{enumerate}
\item UART object/handle
\begin{description}
\item[NOTE:] Pointer to a valid UART object handle from the type \textit{uart\_t}
created with the \textit{UART\_open() function}.
\end{description}
\item Pointer to stop bits
\end{enumerate}
\subsubsection*{Returns}
Returns \textbf{0} on success, or \textbf{-1} on failure.
\subsubsection*{Usage}
\begin{lstlisting}
#include <UART.h>

int stop;

UART_get_stopbits(uart, &stop);
\end{lstlisting}
\subsection{UART\_set\_flowctrl() Function}
The \textit{UART\_set\_flowctrl() function} configures the flow
control from the \textbf{UART} interface.
\subsubsection*{Prototype}
\begin{lstlisting}
#include <UART.h>

int UART_set_flowctrl(uart_t *uart, enum e_flow flow_ctrl);
\end{lstlisting}
\subsubsection*{Arguments}
\begin{enumerate}
\item UART object/handle
\begin{description}
\item[NOTE:] Pointer to a valid UART object handle from the type \textit{uart\_t}
created with the \textit{UART\_open() function}.
\end{description}
\item Flow control Enumeration
\newline
\newline
\begin{tabular}{| c | c |}
\hline
\multicolumn{2}{|c|}{Valid Flow Control} \\
\hline
Enumeration & Flow Control \\
\hline
UART\_FLOW\_NO & None \\
UART\_FLOW\_SW & Software \\
UART\_FLOW\_HW & Hardware \\
\hline
\end{tabular}
\end{enumerate}
\subsubsection*{Returns}
Returns \textbf{0} on success, or \textbf{-1} on failure.
\subsubsection*{Usage}
\begin{lstlisting}
#include <UART.h>

UART_UART_set_flowctrl(uart, UART_FLOW_NO);
\end{lstlisting}
\subsection{UART\_get\_flowctrl() Function}
The \textit{UART\_get\_flowctrl() function} return the flow
control from the \textbf{UART} interface.
\subsubsection*{Prototype}
\begin{lstlisting}
#include <UART.h>

int UART_get_flowctrl(uart_t *uart, int *ret_flow_ctrl);
\end{lstlisting}
\subsubsection*{Arguments}
\begin{enumerate}
\item UART object/handle
\begin{description}
\item[NOTE:] Pointer to a valid UART object handle from the type \textit{uart\_t}
created with the \textit{UART\_open() function}.
\end{description}
\item Pointer to flow control
\end{enumerate}
\subsubsection*{Returns}
Returns \textbf{0} on success, or \textbf{-1} on failure.
\subsubsection*{Usage}
\begin{lstlisting}
#include <UART.h>

int flow;

UART_get_flowctrl(uart, &flow);
\end{lstlisting}
\subsection{UART\_get\_fd() Function (Linux only)}
The \textit{UART\_get\_dev() function} returns the file
descriptor from the \textbf{UART} interface.
\subsubsection*{Prototype}
\begin{lstlisting}
#include <UART.h>

int UART_get_fd(uart_t *uart, int *ret_fd);
\end{lstlisting}
\subsubsection*{Arguments}
\begin{enumerate}
\item UART object/handle
\begin{description}
\item[NOTE:] Pointer to a valid UART object handle from the type \textit{uart\_t}
created with the \textit{UART\_open() function}.
\end{description}
\item Pointer to file descriptor
\end{enumerate}
\subsubsection*{Returns}
Returns \textbf{0} on success, or \textbf{-1} on failure.
\subsubsection*{Usage}
\begin{lstlisting}
#include <UART.h>

int fd;

UART_get_fd(uart, &fd);
\end{lstlisting}
\subsection{UART\_get\_handle() Function (Windows only)}
The \textit{UART\_get\_handle() function} returns the file
handle from the \textbf{UART} interface.
\subsubsection*{Prototype}
\begin{lstlisting}
#include <windows.h>
#include <UART.h>

int UART_get_handle(uart_t *uart, HANDLE *ret_h);
\end{lstlisting}
\subsubsection*{Arguments}
\begin{enumerate}
\item UART object/handle
\begin{description}
\item[NOTE:] Pointer to a valid UART object handle from the type \textit{uart\_t}
created with the \textit{UART\_open() function}.
\end{description}
\item Pointer to file handle
\end{enumerate}
\subsubsection*{Returns}
Returns \textbf{0} on success, or \textbf{-1} on failure.
\subsubsection*{Usage}
\begin{lstlisting}
#include <windows.h>
#include <UART.h>

HANDLE h;

UART_get_handle(uart, &h);
\end{lstlisting}
\subsection{UART\_get\_dev() Function}
The \textit{UART\_get\_dev() function} returns the device
name from the \textbf{UART} interface.
\subsubsection*{Prototype}
\begin{lstlisting}
#include <UART.h>

int UART_get_dev(uart_t *uart, char **ret_dev);
\end{lstlisting}
\subsubsection*{Arguments}
\begin{enumerate}
\item UART object/handle
\begin{description}
\item[NOTE:] Pointer to a valid UART object handle from the type \textit{uart\_t}
created with the \textit{UART\_open() function}.
\end{description}
\item Pointer to device name
\end{enumerate}
\subsubsection*{Returns}
Returns \textbf{0} on success, or \textbf{-1} on failure.
\subsubsection*{Usage}
\begin{lstlisting}
#include <UART.h>

char *dev;

UART_get_dev(uart, &dev);
\end{lstlisting}
\section{Miscellaneous Functions}
This section contains generic miscellaneous functions such as
getting the bytes available on the receiver from the \textbf{UART}
interface and the information according the library like the
library name and the library version.
\subsection{UART\_get\_bytes\_available() Function}
The \textit{UART\_get\_bytes\_available() function} returns the number of
bytes available in the receive channel.
\subsubsection*{Prototype}
\begin{lstlisting}
#include <UART.h>

int UART_get_bytes_available(uart_t *uart, int *ret_num);
\end{lstlisting}
\subsubsection*{Arguments}
\begin{enumerate}
\item UART object/handle
\begin{description}
\item[NOTE:] Pointer to a valid UART object handle from the type \textit{uart\_t}
created with the \textit{UART\_open() function}.
\end{description}
\item Pointer to bytes
\end{enumerate}
\subsubsection*{Returns}
Returns \textbf{0} on success, or \textbf{-1} on failure.
\subsubsection*{Usage}
\begin{lstlisting}
#include <UART.h>

int bytes;

UART_get_bytes_available(uart, &bytes);
\end{lstlisting}
\subsection{UART\_get\_libname() Function}
The \textit{UART\_get\_libname() function} returns the library
name string.
\subsubsection*{Prototype}
\begin{lstlisting}
#include <UART.h>

char *UART_get_libname(void);
\end{lstlisting}
\subsubsection*{Arguments}
None
\subsubsection*{Returns}
Returns the library name string.
\subsubsection*{Usage}
\begin{lstlisting}
#include <UART.h>

printf("%s", UART_get_libname());
\end{lstlisting}
\subsection{UART\_get\_libversion() Function}
The \textit{UART\_get\_libversion() function} returns the library
version string.
\subsubsection*{Prototype}
\begin{lstlisting}
#include <UART.h>

char *UART_get_libversion(void);
\end{lstlisting}
\subsubsection*{Arguments}
None
\subsubsection*{Returns}
Returns the library version string.
\subsubsection*{Usage}
\begin{lstlisting}
#include <UART.h>

printf("%s", UART_get_libversion());
\end{lstlisting}
\end{document}
